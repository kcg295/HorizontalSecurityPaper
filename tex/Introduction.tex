We start our journey by imagining an online community of bakers, passionate
about creating delicious cakes. This community maintains an online repository of
recipes to make delicious cakes, which are contributed by their members. In
order to maintain the community, the community decided to elect their moderators
and their head administrator. When a new head administrator is elected the old
head administrator steps down and transfers administrator power to the newly
elected member. This is done by editing the administrator group to first add the
newly elected member, then editing it again to remove themselves. This process
had occurred successfully 5 times in the past, so the community feels that the
system works.

The current election has three major candidates among the members in the
community: Ronald, the current head administrator, Ernie, a well respected
community member, and Jim, former assistant administrator. After a long election
cycle of debating and discussing cakes, what makes a good recipe, and whether
or not members are allowed to post Amazon affiliate links, the community decides
that Jim should be their new head administrator. With the decision made, a date
is chosen for Ronald to hand over power to Jim.

However, Roland does not think Jim is qualified to be the head administrator. In
Ronald's mind, he and only he knows what makes a good cake, what makes a good
recipe, and what is best for the community. He resents that the community chose
Jim, and decides not to give up power. Instead of editing the administrator
group to add Jim, Ronald edits the member list to remove Jim, Ernie, and their
most vocal supporters. Worse still, Ronald decides to remove many recipes of
cakes that he does not like from the online repository. He then closes the
community to new members and punishes people who publish recipes he does not
approve of by removing their ability to post for three weeks. Elections for new
administrators go on for a while, but supporters of Ronald's opponents are
removed from the community right before the elections occur. After a long streak
of winning many head administrator elections, Ronald feels that there is no
point to continuing the elections and stops holding them completely.

Posts in the community begin to change. In the fear of being removed from the
community or being silenced for three weeks, members go out of their way to post
only recipes that Ronald would like. Experimentation with ingredients that
Ronald is allergic to ceases. New takes on Ronald's favorite cake, carrot cake,
become the highest priority of those who are still active. What was once a
vibrant community has dwindled in number, and the vast array of cakes recipes
have been trimmed down to a select few.

What went wrong? What caused this to occur? We know that the decline occurred
when Ronald decided not to transfer power, but what was the vulnerability that
allowed this to happen? To discover this, we need to analyze the ways in which
the community decided to run.

First, we need to recognize that this community is broken into two separate
power structures. First is the governance structure, or the process in which
the community decides to organize itself socially. The structure they chose is
intended to be bottom-up, at least partially. In other words, the power
originates in the members, not a hierarchical figure. This is recognized through
the process of elections, where the community itself decides who will continue
to lead.

The second power structure is access control to the resources of the community.
While many community members may contribute to the recipe list, while new
members may join the community, and while members themselves may vote for the
new head administrator, the decision of whether or not these added recipes
remain, the new members are allowed, or the transfer of power takes place is
made in an entirely top-down, or hierarchical way. The head administrator makes
these decisions, and has full control over the technology to enforce these
decisions, or not to enforce them, as the administrator chooses.

As we can see, there is a mismatch in the goals of these two structures. While
one seeks to be bottom-up and respect sovereignty of the community, the other is
top-down and respects only the word of the ruler. Thus, when the ruler states
that the election is invalid and removes their opponents from the community, the
access control system allows this.

Why? Why must access control systems be so centralized, so hierarchical, in
their design. Much like Marx broke society down into the base and the
superstructure, with the base being production and the superstructure being the
society that is built on top of it,~\cite{marx1911contribution} we too can break
down digital communities into a base and superstructure, with a base being the
technologies that are used to maintain the community and its resources and the
superstructure being the communities that rely on those technologies. Much as
Marx said ``the windmill gives you society with the feudal lord; the steam mill,
society with the industrial capitalist,'' \cite{marx1920poverty} we can add
``hierarchical access control gives you a society under the control of the
single head administrator.'' In other words, so long as we use non-democratic
forms of moderating and administering our technology, we can never achieve a
truly democratic digital society.
